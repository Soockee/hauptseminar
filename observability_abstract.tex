%%
%% This is file `sample-manuscript.tex',
%% generated with the docstrip utility.
%%
%% The original source files were:
%%
%% samples.dtx  (with options: `manuscript')
%% 
%% IMPORTANT NOTICE:
%% 
%% For the copyright see the source file.
%% 
%% Any modified versions of this file must be renamed
%% with new filenames distinct from sample-manuscript.tex.
%% 
%% For distribution of the original source see the terms
%% for copying and modification in the file samples.dtx.
%% 
%% This generated file may be distributed as long as the
%% original source files, as listed above, are part of the
%% same distribution. (The sources need not necessarily be
%% in the same archive or directory.)
%%
%% The first command in your LaTeX source must be the \documentclass command. This is the generic manuscript mode required for submission and peer review.
\documentclass[11pt, manuscript, screen, nonacm=true, natbib=true]{acmart-backref}
%%
%% \BibTeX command to typeset BibTeX logo in the docs
\AtBeginDocument{%
  \providecommand\BibTeX{{%
    \normalfont B\kern-0.5em{\scshape i\kern-0.25em b}\kern-0.8em\TeX}}}

%% The majority of ACM publications use numbered citations and
%% references.  The command \citestyle{authoryear} switches to the
%% "author year" style.
%%
%% If you are preparing content for an event
%% sponsored by ACM SIGGRAPH, you must use the "author year" style of
%% citations and references.
%% Uncommenting
%% the next command will enable that style.
%%\citestyle{acmauthoryear}

%%
%% end of the preamble, start of the body of the document source.
\begin{document}

%%
%% The "title" command has an optional parameter,
%% allowing the author to define a "short title" to be used in page headers.
\title{ A view on the evolution of observability in distributed systems }

%%
%% The "author" command and its associated commands are used to define
%% the authors and their affiliations.
%% Of note is the shared affiliation of the first two authors, and the
%% "authornote" and "authornotemark" commands
%% used to denote shared contribution to the research.
\author{Simon Stockhause}



%%
%% The abstract is a short summary of the work to be presented in the
%% article.
%\begin{abstract}

  
%\end{abstract}

%%
%% This command processes the author and affiliation and title
%% information and builds the first part of the formatted document.
\maketitle

\section{Abstract}
 Observability takes an essential role in modern distributed systems. The complexity of distributed system is found in many different layers. The classical view of observability consist of monitoring hardware, which run the application. This classical view is not sufficient in modern distributed environments such as cloud-native applications. In order to understand the observability needs of today, the transition between these two approaches for System development and System administration have to be compared. Many real-world examples of those transitions exist, which shall be studied to extract valuable lessons.\cite{Thingbo2016}\cite{Hedman2016}\cite{GHOLAMI201631} The need for concepts and technologies to observe distributed systems on every possible layer are as old as distributed systems itself. In context of modern distributed systems the term observability rose up around 2013, but has its origin in the 1960s.\cite{KALMAN1960491} Some of the subjects such as metrics, logs and traces are much older then the modern term. This paper aims to provide an view on the evolution of observability, in which the most important observability-subjects are analysed. The analysis shall consider the chronological evolution of each subject and its place in distributed system in its time. Furthermore the analysis shall give an understanding of the origin of the current state of the art of each subject. The difference between self-hosted and cloud-based system observability and administration shall be discussed. Especially the agility of developer and operators in each context has to be evaluated. The goal of this paper consists of giving an overview of the history of observability and its implications for current observability standards in cloud environments.
\section{Summary Draft}
\begin{itemize}
\item Introduction
\item Origin of Observability
\begin{itemize}
    \item Observability in control theory
    \item Nasa moon?
    \item First distributed applications. Arpanet/email/early clientserver
\end{itemize}
\item Monitoring
\begin{itemize}
    \item Monitoring vs. Observability
    \item Transition to the cloud
\end{itemize}
\item Modern Observability Frameworks by example
\begin{itemize}
    \item Elastic-Stack
    \item Grafana, Prometheus
    \item Zipkin, Jaeger
    \item OpenTelemtry
    \item AWS Services / Cloudprovider Services
\end{itemize}
\item Conclusion
\end{itemize}

%%
%% The next two lines define the bibliography style to be used, and
%% the bibliography file.
\bibliographystyle{ACM-Reference-Format}
\bibliography{base}

%%
%% If your work has an appendix, this is the place to put it.
\appendix

\end{document}
\endinput
%%
%% End of file `sample-manuscript.tex'.
